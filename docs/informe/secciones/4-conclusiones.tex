\section{Conclusiones}

    Como conclusión del trabajo, y adoptando una visión general sobre los experimentos realizados, es posible afirmar que ninguno de los métodos evaluados resulta superior a los demás en todos los escenarios. La elección de uno u otro deberá ser evaluada cuidadosamente a partir de las características propias del video a procesar, y teniendo en cuenta si busca optimizarse el tiempo de ejecución o la calidad de los resultados obtenidos. En muchas ocasiones, puede resultar conveniente combinar dos de los métodos; por ejemplo, en los momentos en que hay cortes de escena, podría ser deseable aplicar el método ``vecino más cercano'', a pesar de que se utilice uno de los otros dos para el resto del video, ya que, como pudo observarse, el primero hace un mejor trabajo en ese tipo de situaciones. No obstante, para ello sería necesario disponer de un método para reconocer estos cambios de escena, lo cual, en determinados contextos, puede resultar inviable.

    Es de interés notar que, a pesar de que las hipótesis iniciales no lo preveían, el método de interpolación lineal produjo resultados de una calidad considerable, aproximándose más en las métricas a los videos reales que el de interpolación por splines, produciendo en general \emph{artifacts} menos pronunciados y teniendo al mismo tiempo un mejor rendimiento temporal y una mayor facilidad de implementación. Hay que tener en cuenta, sin embargo, que los videos generados por el método de \emph{splines} resultan más suaves a la vista, siendo menos perceptibles que en los videos generados mediante interpolación lineal los puntos de quiebre marcados por los cuadros no interpolados. Aquí, otra vez, se hace necesario un análisis particular del caso de aplicación para decidir cuál de los métodos es conveniente aplicar.
